\documentclass[11pt]{article}
\usepackage[hidelinks]{hyperref}
\usepackage[letterpaper]{geometry}
\geometry{verbose,tmargin=1in,bmargin=1in,lmargin=1in,rmargin=1in}
\usepackage[acronym,nomain,nonumberlist,nogroupskip,nopostdot]{glossaries} % for glossary of acronyms
\usepackage{siunitx}
\usepackage{verbatim}
\usepackage{booktabs}
\usepackage{multirow}
\usepackage{threeparttable}
\usepackage{longtable}
\usepackage{rotating}
\usepackage{pdflscape}
\usepackage{cancel}
\usepackage{mathrsfs}
\usepackage{mhchem}
\setcounter{tocdepth}{4}
\setcounter{secnumdepth}{4}
\usepackage{xcolor}
\usepackage{amsmath,mathtools}

% \setacronymstyle{long-short}
% \loadglsentries{/Users/anovak/projects/tex_inputs/glossary}

\newcommand{\mdash}{\discretionary{}{}{\kern 0.1em}---\discretionary{}{}{\kern 0.1em}}
\newcommand\Tstrut{\rule{0pt}{2.6ex}}         % = `top' strut
\newcommand\Bstrut{\rule[-0.9ex]{0pt}{0pt}}   % = `bottom' strut
\newcommand{\beq}{\begin{equation}\begin{aligned}}
\newcommand{\eeq}{\end{aligned}\end{equation}}
\newcommand{\beqs}{\begin{equation*}\begin{aligned}}
\newcommand{\eeqs}{\end{aligned}\end{equation*}}

\usepackage{color} % \textcolor
\usepackage{soul} % \hl
\newcommand{\unfinished}[1][UNFINISHED]{\textcolor{red}{\textbf{\hl{#1}}}}

\usepackage{tcolorbox}

\setlength{\parskip}{5pt}

\begin{document}
\noindent {\bf \Large NPRE 455: Homework 1}\\

%\todo{add python refresher}
%\todo{add sympy root finding}

\begin{tcolorbox}
This homework will freshen up your math skills by asking you to solve several differential equations we will frequently encounter in this class. It is {\it crucial} that you do well on this assignment in order to prepare you for this course. On Canvas, please see the ``Differential Equations Review'' file in the ``Handouts'' folder if you need a refresher on differential equations.\\

For each problem, the high-level learning goal is indicated in bold font preceding the problem statement.
\end{tcolorbox}

\vspace{2cm} 

\noindent 1. ({\bf factual concept recall}) Provide a {\it short} (1-2 sentences) answer to each of the following questions.

\begin{enumerate}
\item[(a)] ({\bf 1 point}) Define ``binding energy.''
\begin{enumerate}
    \item Binding energy can be thought of as the energy it would take to separate a nucleus into every individual nucleon.
\end{enumerate}
\item[(b)] ({\bf 2 points}) Define a decay constant, $\lambda$. What are its units?
\begin{enumerate}
    \item A decay constant $\lambda$ (typical units $\frac{1}{s}$) describes how quickly a kind of atom decays, such that after time $t$, $exp(t/\lambda)$ is the likelihood of a given atom, or the fraction of a given macroscopic quantity, remaining.
\end{enumerate}
\item[(c)] ({\bf 2 points}) Define a microscopic cross section, $\sigma_i$. What are its units?
\begin{enumerate}
    \item A microscopic cross section $\sigma_i$  (units $b$, $10^{-28}m^2$), describes the feasibility of a particle (here, a neutron) colliding with a target (nucleus) to trigger the phenomenon $i$.
    Dimensionally, it is an area facing the oncoming particle, but it is not literally the cross sectional area of the target.
\end{enumerate}
\item[(d)] ({\bf 1 point}) Define a macroscopic cross section, $\Sigma_i$. What are its units?
\begin{enumerate}
    \item The macroscopic cross section $\Sigma_i$ (units $\frac{1}{m}$) of a material describes how quickly (over distance, not time) a particle will interact with the material in fashion $i$, on average.
    Its inverse, the mean free path, is the average distance the particle would travel before interacting.
\end{enumerate}
\item[(e)] ({\bf 1 point}) The total cross section, $\sigma_t$, can always be written as the sum of two other cross sections, scattering ($\sigma_s$) and absorption ($\sigma_a$).

\beq
\sigma_t\equiv \sigma_s+\sigma_a
\eeq

\item[(f)] ({\bf 1 point}) What is a ``resonance'' in plots of $\sigma$? 
\begin{itemize}
    \item A ``resonance'' is a peak in such a plot at a specific energy, for which $\sigma$ is drastically higher than other nearby energies.
    \item This might be, for example, a low energy such that the corresponding wavelength lines up with the length of a crystal lattice bond.
\end{itemize}
\item[(g)] ({\bf 1 point}) Below what energy do neutrons typically start interacting with entire molecules, clumps of atoms, and material lattice structures?
\begin{itemize}
    \item Below around 1eV, when the corresponding wavelength grows to the scale of the distance between atoms (or a specific ``Bragg Cutoff'' energy associated with the spacings of a lattice), the neutron may interact with entire molecules or crystals.
\end{itemize}
\item[(h)] ({\bf 1 point}) What does the symbol $\nu$ represent? What is a typical value?
\begin{itemize}
    % \item The speed of a neutron, $\nu$, varies wildly but might be millions of miles per hour, or at the low end (0.01eV), still 980 miles per hour, faster than a 747.
    \item $\nu$, the number of neutrons released in a fission event (either promptly or with delay), and an average value $\bar{\nu}$ might be around 2.4 for plutonium.
\end{itemize}
\item[(i)] ({\bf 1 point}) Why do resonances seem to suddenly ``stop'' on plots of $\sigma$?
\begin{itemize}
    \item Beyond a certain point, resonances are so closely packed that it becomes unfeasible to manually tell, and instead a probability table is used of the likelihood of it resonating. At even higher energy, the tables are no longer needed as the resonances fully overlap.
\end{itemize}
\item[(j)] ({\bf 1 point}) What is the difference between fissile and fissionable nuclides? Give examples of each.
\begin{itemize}
    \item Fissile nuclei, like $\mathrm{^{235}U}$, readily undergo fission when struck by a thermal neutron. Fissionable nuclides, like $\mathrm{^{238}U}$, are more loosely defined but are more known for fission triggered by high energy neutrons.
\end{itemize}
\item[(k)] ({\bf 1 point}) What does the symbol $\beta$ represent? What is a typical value?
\begin{itemize}
    \item $\beta$ represents the fraction of fission neutrons which are delayed, $\bar{\nu}_d/\bar{\nu}$, and is typically very small, 0.0065 for $\mathrm{^{235}U}$ in thermal fission.
\end{itemize}
\item[(l)] ({\bf 1 point}) What does the symbol $\chi_p$ represent? 
\begin{itemize}
    \item $\chi$ is a probability density function describing the energies a neutron will be born at; $\chi_p$ represents this distribution for prompt neutrons.
\end{itemize}
\item[(m)] ({\bf 1 point}) How much energy is released in a typical fission reaction?
\begin{itemize}
    \item A typical fission reaction will release around 193MeV of energy. (Discounting neutrinos, since those are essentially unrecoverable)
\end{itemize}
\end{enumerate}

\clearpage
\noindent 2. ({\bf how to simplify and solve spherical Helmholtz equations}) $\phi(r)$ is governed by the following differential equation,

\beq
\label{eq:one}
\frac{1}{r^2}\left\lbrack \frac{d}{dr}\left(r^2\frac{d\phi(r)}{dr}\right)\right\rbrack \pm\gamma^2\phi(r)=0
\eeq

\noindent This type of equation, of the form $\nabla^2\phi\pm\gamma^2\phi=0$, is also called a ``Helmholtz equation'' because it appears so often in engineering -- Eq. \eqref{eq:one} is identical to the neutron diffusion equation in spherical coordinate systems (under certain circumstances). 

\begin{enumerate}
\item[(a)] ({\bf 5 points}) By letting $\phi(r)\equiv\mu(r)/r$, derive the differential equation for $\mu(r)$. Leave the $\pm$ in your answer.\\
    \begin{align*}
    \frac{1}{r^2}\left\lbrack \frac{d}{dr}\left(r^2\frac{d\phi(r)}{dr}\right) \right\rbrack &\pm\gamma^2\phi(r)=0 \\  
    \frac{1}{r^2}\left\lbrack \frac{d}{dr}\left(r^2\frac{d}{dr}\left[\frac{\mu(r)}{r}\right]\right) \right\rbrack &\pm\gamma^2\left(\frac{\mu(r)}{r}\right)=0 \\
    \frac{1}{r^2}\left\lbrack \frac{d}{dr}\left(r^2\left[\frac{\mu'(r)r - \mu(r)}{r^2}\right]\right) \right\rbrack &\pm\gamma^2\left(\frac{\mu(r)}{r}\right)=0 \\
    \frac{1}{r^2}\left\lbrack \frac{d}{dr}\left(\mu'(r)r - \mu(r)\right) \right\rbrack &\pm\gamma^2\left(\frac{\mu(r)}{r}\right)=0 \\
    \frac{1}{r^2}\left\lbrack \mu''(r)r + \mu'(r) - \mu'(r) \right\rbrack &\pm\gamma^2\left(\frac{\mu(r)}{r}\right)=0 \\
    \frac{\mu''(r)}{r} &\pm\gamma^2\left(\frac{\mu(r)}{r}\right)=0 \\
    \mu''(r) &\pm\gamma^2\mu(r) = 0
    \end{align*}
    % \begin{align*}
    %     \frac{1}{r^2}\left\lbrack \frac{d}{dr}\left(r^2\frac{d\phi(r)}{dr}\right) \right\rbrack& \pm\gamma^2\phi(r)=0
    % \end{align*}
\item[(b)] ({\bf 10 points}) Using the solutions for $\mu(r)$, write down the general solutions for $\phi(r)$ assuming the $+$ sign.
    \begin{align*}
        \mu''(r) + \gamma^2\mu(r) &= 0 \\ 
        \rightarrow\mu(r) &= C_1cos(\gamma r) + C_2sin(\gamma r)\\
        \phi(r) &= rC_1cos(\gamma r) + rC_2sin(\gamma r)\\
    \end{align*}
\item[(c)] ({\bf 10 points}) Using the solutions for $\mu(r)$, write down the general solutions for $\phi(r)$ assuming the $-$ sign.
    \begin{align*}
        \mu''(r) - \gamma^2\mu(r) &= 0 \\ 
        \rightarrow\mu(r) &= C_1cosh(\gamma r) + C_2sinh(\gamma r)\\
        \phi(r) &= rC_1cosh(\gamma r) + rC_2sinh(\gamma r)\\
    \end{align*}
\end{enumerate}

\clearpage
\noindent 3. ({\bf how to solve first-order ODEs}) Radioactive decay, for a nuclide species $N(t)$ [atoms/m$^3$] and decay constant $\lambda$ [1/s] is governed by the following differential equation,

\beq
\label{eq:one1}
\frac{dN(t)}{dt}+\lambda N(t)=f(t)
\eeq

\noindent where $f(t)$ is a function [atoms/m$^3$/s] which includes any additional rates of production and removal of the nuclide species $N$ (e.g., production as a fission product, loss from neutron capture, ...).

\begin{enumerate}
\item[(a)] ({\bf 10 points}) Solve Eq. \eqref{eq:one1} subject to the boundary condition $N(0)=N_0$.
    Let
    \begin{equation}
    p(t) = exp(\int_{0}^{t}\lambda du) = e^{\lambda t}
    \end{equation}
    then
    \begin{equation}
    \frac{dp(t)}{dt} = \lambda e^{\lambda t}
    \end{equation}
    Multiplying by $p(x)$,
    \begin{align*}
        \frac{dN(t)}{dt}e^{\lambda t}+\lambda N(t)e^{\lambda t}=f(t)e^{\lambda t} \\
        \frac{d}{dt}\left[N(t)e^{\lambda t}\right]=f(t)e^{\lambda t} \\   
        N(t)e^{\lambda t} = \int f(t)e^{\lambda t}dt + C \\
        N(t) = e^{-\lambda t} \left(\int f(t)e^{\lambda t}dt + C \right)\\
        % N(t)e^{\lambda t} = \int_{0}^{t} f(u)e^{\lambda u}du \\
        % N(t) = e^{-\lambda t} \int_{0}^{t} f(u)e^{\lambda u}du\\
    \end{align*}
    Applying the boundary condition $N(0) = N_0$,
    % \begin{align*}
    %     N(t) = e^{-\lambda t} \int_{0}^{t} f(u)e^{\lambda u}du\\
    % \end{align*}
    \begin{align*}
        N(0) &= e^{-\lambda 0} \left(\left[\int f(t)e^{\lambda t}dt\right]_0 + C \right) = N_0 \\
        N_0 &= \left[\int f(t)e^{\lambda t}dt\right]_0 + C \\
        C &= N_0 - \left[\int f(t)e^{\lambda t}dt\right]_0\\
        N(t) &= e^{-\lambda t} \left(\int f(t)e^{\lambda t}dt + N_0 - \left[\int f(t)e^{\lambda t}dt\right]_0 \right)\\
        N(t) &= e^{-\lambda t} \left(\int_{0}^{t} f(u)e^{\lambda u}du + N_0 \right)\\
    \end{align*}
\end{enumerate}


\clearpage
\noindent 4. ({\bf solving Cartesian Helmholtz equations}) Consider the following differential equation -- under certain circumstances, this equation is identical to the neutron diffusion equation in Cartesian geometries.

\beq
\label{eq:one2}
\frac{d^2\phi(x)}{dx^2}+a^2\phi(x)=0
\eeq

\begin{enumerate}
\item[(a)] ({\bf 10 points}) Solve Eq. \eqref{eq:one2} subject to the boundary conditions $\phi(0)=A$ and $\phi(L)=B$.\\
\begin{itemize}
    \item As a predicted general solution to Helmholtz equations, let $\phi(x)=C_1cos(ax) + C_2sin(ax)$
    \begin{align*}
        \frac{d^2\phi(x)}{dx^2}+a^2\phi(x)&=0\\
        -\left(a^2C_1cos(ax) + a^2C_2sin(ax)\right) + a^2 (C_1cos(ax) + C_2sin(ax)) &= 0\\
        0 &= 0\\
    \end{align*}
    Then,
    \begin{align*}
        \phi(x)&=C_1cos(ax) + C_2sin(ax) \\
        \phi(0)&=A \\
        A &=C_1cos(0) + C_2sin(0) \\
        C_1 &= A \\
    \end{align*}
    \begin{align*}
        \phi(L)&=B \\
        B &= Acos(aL) + C_2sin(aL) \\
        B - Acos(aL) &= C_2sin(aL) \\
        C_2 &= \frac{B-Acos(aL)}{sin(aL)} \\
    \end{align*}
    \begin{equation}
        \phi(x) = Acos(ax) + \frac{B-Acos(aL)}{sin(aL)}sin(ax)
    \end{equation}

\end{itemize}

\end{enumerate}

\clearpage
\noindent 5. ({\bf how to solve inhomogeneous ODEs}) Consider the following differential equation -- under certain circumstances, this equation is identical to the neutron diffusion equation in Cartesian geometries.

\beq
\label{eq:one3}
\frac{d^2\phi(x)}{dx^2}-a^2\phi(x)=2x^3
\eeq

\noindent This equation is called ``inhomogeneous'' because not every term in the differential equation contains $\phi$, our unknown.

\begin{enumerate}
\item[(a)] ({\bf 10 points}) Solve Eq. \eqref{eq:one3} subject to the boundary conditions $\phi(0)=A$ and $\phi(L)=B$.
\end{enumerate}

\clearpage
\noindent 6. ({\bf deep concept understanding}) For each of the following statements, indicate whether the statement is true/false or yes/no. Provide an explanation.

\begin{enumerate}
\item [(a)] ({\bf 5 points}) (True/False) Kinetic energy is always conserved before and after a reaction. 
\item [(b)] ({\bf 5 points}) (Yes/No) Very slow neutrons are traveling through water and yttrium hydride; in order to study this process, you first need to compute the macroscopic cross sections for these two materials. You look up the cross sections for the individual atoms and then combine them as follows.

\beq
\Sigma_t [H_2O]=\sigma_{H,t}N_H+\sigma_{O,t}N_O
\eeq

\beq
\Sigma_t [YH_2]=\sigma_{H,t}N_H+\sigma_{Y,t}N_Y
\eeq

Is this approach correct?
\end{enumerate}

\clearpage
\noindent 7. ({\bf $\Sigma$ for mixtures, terminology definitions}) One way that tritium is produced in a nuclear reactor is from an $(n,\gamma)$ reaction on deuterium,

\beq
\ce{^2_1 H} +n\rightarrow\ce{^3_1 H}+\gamma
\eeq

\begin{enumerate}
\item[(a)] ({\bf 20 points}) Predict the tritium production rate, in units of mg/year, from the above reaction in a light water reactor. Assume that macroscopic cross sections are not a function of time. Parameters describing the reactor are provided below. Report your answer to THREE places after the decimal point.

\begin{itemize}
\item Average thermal neutron flux: $\phi=10^{13}$ 1/cm$^2$/s
\item Average water density: 750 kg/m$^3$
\item Volume occupied by water: 9.54 \si{\cubic\meter}
\item $\ce{^2_1 H}$ atomic concentration in elemental (natural) hydrogen: 0.015\%
\item Thermal neutron capture cross section for $\ce{^2_1 H}$: 506 $\mu$b
\end{itemize}
\end{enumerate}

({\it Hint: Be very careful with units. The correct answer is within the range 0.03--0.06 mg/year.}).

\end{document}

