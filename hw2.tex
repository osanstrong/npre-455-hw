\documentclass[11pt]{article}
\usepackage[hidelinks]{hyperref}
\usepackage[letterpaper]{geometry}
\geometry{verbose,tmargin=1in,bmargin=1in,lmargin=1in,rmargin=1in}
\usepackage[acronym,nomain,nonumberlist,nogroupskip,nopostdot]{glossaries} % for glossary of acronyms
\usepackage{siunitx}
\usepackage{verbatim}
\usepackage{booktabs}
\usepackage{multirow}
\usepackage{threeparttable}
\usepackage{longtable}
\usepackage{rotating}
\usepackage{pdflscape}
\usepackage{cancel}
\usepackage{mathrsfs}
\usepackage{mhchem}
\setcounter{tocdepth}{4}
\setcounter{secnumdepth}{4}
\usepackage{xcolor}
\usepackage{amsmath,mathtools}
\usepackage{hyperref} % Link to desmos page

\setacronymstyle{long-short}

\newcommand{\mdash}{\discretionary{}{}{\kern 0.1em}---\discretionary{}{}{\kern 0.1em}}
\newcommand\Tstrut{\rule{0pt}{2.6ex}}         % = `top' strut
\newcommand\Bstrut{\rule[-0.9ex]{0pt}{0pt}}   % = `bottom' strut
\newcommand{\beq}{\begin{equation}\begin{aligned}}
\newcommand{\eeq}{\end{aligned}\end{equation}}
\newcommand{\beqs}{\begin{equation*}\begin{aligned}}
\newcommand{\eeqs}{\end{aligned}\end{equation*}}
\usepackage{tcolorbox}
\setlength{\parskip}{5pt}

\begin{document}
\noindent {\bf \Large NPRE 455: Homework 2}\\

\noindent 1. ({\bf factual concept recall}) Provide a {\it short} (1-2 sentences) answer to each of the following questions.

\begin{enumerate}
\item[(a)] ({\bf 1 point}) What is ``breeding?'' Give an example using \ce{^{232}Th}.
\item[(b)] ({\bf 2 points}) What is the difference between a ``multiplying'' system and a ``non-multiplying'' system? Give a real-life example of each.
\item[(c)] ({\bf 1 point}) What is the name of the evaluated data library developed in the United States?
\item[(d)] ({\bf 4 points}) What do the four symbols in the four-factor formula represent? Define each one and give a short physical interpretation.
\item[(e)] ({\bf 2 points}) What is meant by ``spatial self-shielding?''
\item[(f)] ({\bf 2 points}) Why can it be a poor design choice for a reactor to be over-moderated?
\end{enumerate}

\clearpage
\noindent 2. ({\bf reaction rates and definitions}) Under certain circumstances, the neutron flux in a sphere of radius $R$ is given as 

\beq
\label{eq:flux}
\phi(r)=\frac{\phi_0R\sin{\left(\frac{\pi r}{R}\right)}}{r}
\eeq

where $\phi_0=13.1\times10^{13}$ [1/cm$^2$/s], $R=50$ [cm], $\Sigma_a=0.108$ [1/cm], $\Sigma_s=0.21$ [1/cm], and $\Sigma_f=0.0727$ [1/cm].

\begin{enumerate}
\item[(a)] ({\bf 3 points}) What is the maximum flux in the sphere?
\item[(b)] ({\bf 3 points}) Plot $\phi(r)$ as a function of $r$. This should be a computer-generated plot, not a hand sketch. Include labels for your axes with units.
\item[(c)] ({\bf 4 points}) What is the total number of neutrons in the sphere at any given time? You may assume all neutrons are at an energy of 0.025 eV.
%\item[(d)] ({\bf 4 points}) Estimate the number of neutrons at any given time in 1 cm$^3$ at the center of the sphere, and the number of fissions taking place in 1 second in that volume. ({\it Hint}: what is the radius of this 1 cm$^3$ sphere, and how much does flux from 1(b) vary over that region?)
%\item[(e)] ({\bf 4 points}) Estimate the rate of energy (W) generated due to fission in a sphere of volume 1 cm$^3$ at the center of the sphere.
\item[(d)] ({\bf 4 points}) How much power (W) is generated due to fission in the entire sphere?
\item[(e)] ({\bf 5 points}) At some time later, the power level of the reactor changes. Assuming the flux profile is still governed by Eq. \eqref{eq:flux}, derive the relationship between $\phi_0$ and the power $P$. 
\item[(f)] ({\bf 2 points}) What is the average distance that a neutron travels between collisions? Find it numerically.
\end{enumerate}

\clearpage
\noindent 3. ({\bf deep concept understanding}) For each of the following statements, indicate whether the statement is true or false. Provide an explanation.

\begin{enumerate}
\item [(a)] ({\bf 5 points}) (True/False) A reactor operating at a steady-state power $P$ can have $k<1$.
\begin{itemize}
    \item \textbf{True}, but only by caveat. If the reactor had a multiplication value $k<1$, then every fission incident will trigger less than one other incident on average, and not sustain a steady reaction. This could be supplemented by an external source which introduced neutrons, thus creating some non-zero power steady state equilibrium.
\end{itemize}
\item [(b)] ({\bf 5 points}) (True/False) A reactor operating at a steady-state power $P$ can have $k>1$.
\begin{itemize}
    \item \textbf{False}. If just the nebulous $k_\infty$ were $>1$, then leakage might reduce $k$ to around exactly 1. However, without some other factor to lose neutrons, or trigger fewer fusion events per neutron (which would have already been factored into $k$), then this means that $n_t$ fission events at every moment in time are always triggering $n_{t+}>n_t$ fission events in the future, increasing the future power of the reactor. You'd need to suppose a neutron sink that weren't proportional to how many neutrons there are (because this is already factored into $k$), like a magic wand that deletes $10^5$ neutrons per cubic centimeter per second or something.
\end{itemize}
\item [(c)] ({\bf 5 points}) (True/False) The non-leakage probability, $P_{nl}$, is only a function of geometry.
\begin{itemize}
    \item \textbf{False.} The probability of leakage also depends on the energy distribution itself, and since the probability of what neutrons at what energies leak then goes back to affect multiplication, there's a cyclic dependency such that you can't \emph{practically} separate these factors, only conceptually. 
\end{itemize}
\item [(d)] ({\bf 5 points}) (True/False) Consider a sphere of UO$_2$ initially surrounded by vacuum. You enclose the sphere in a layer of \ce{^{157}Gd}, a strong neutron absorber. $k$ should increase.
\begin{itemize}
    \item \textbf{True}, especially if the vacuum were infinite to begin with. Even if most of the neutrons that interact with the layer of \ce{^{157}Gd} were absorbed, \emph{all} of the neutrons that went into the vacuum were already \emph{entirely} lost, so any neutrons that are scattered back into the UO$_2$ still increases the multiplication rate $k$.
\end{itemize}
\end{enumerate}

\clearpage
\noindent 4. ({\bf flux energy spectra}) Download the file {\tt flux\_spectrum1.csv} from Canvas (in Files/Homeworks/HW2). This file contains, in CSV (comma-separated value) format, a neutron energy spectrum in a fission reactor. On each line of this file, the data is formatted as {\tt Emin, Emax, flux}, where {\tt Emin} represents the lower bound of an energy bin, {\tt Emax} represents the upper bound of an energy bin, and {\tt flux} is the measured flux in units of 1/cm$^2$/s.
\begin{figure}[htb!]
\centering
\includegraphics[width=0.6\linewidth]{figures/hw2_spec1_bw.png}
\caption{The flux spectrum given in file \texttt{flux\_spectrum1.csv}, normalized by bin width.}
\label{fig:spec1_bw}
\end{figure}
\begin{figure}[htb!]
\centering
\includegraphics[width=0.6\linewidth]{figures/hw2_spec1_ul.png}
\caption{The flux spectrum given in file \texttt{flux\_spectrum1.csv}, normalized per unit lethargy.}
\label{fig:spec1_ul}
\end{figure}    
\begin{enumerate}
\item [(a)] ({\bf 5 points}) Generate two different plots (computer-generated) of this data, both on log-log axes. These two plots should show (i) flux, normalized by the bin widths {\tt Emax - Emin}, as a function of energy; and (ii) flux, normalized per unit lethargy $\ln{(\text{\tt Emax/Emin})}$, as a function of energy. Include labels on your axes.

\begin{itemize}
    \item Figures~\ref{fig:spec1_bw} and \ref{fig:spec1_ul} show the flux spectrum given in file \texttt{flux\_spectrum1.csv}, as normalized by bin width and by unit lethargy, respectively.
\end{itemize}
\item [(b)] ({\bf 3 points}) Describe (qualitatively) the physical process occurring at the large negative spikes in the spectrum. What phrase do we use to describe this phenomenon?
\begin{itemize}
    \item At the large negative spikes, the cross section of certain materials is so high that neutrons around this energy are almost immediately absorbed or scattered to a different energy. This phenomenon is called ``energy self-shielding.''
\end{itemize}
\item [(c)] ({\bf 2 points}) Is this a thermal-spectrum or fast-spectrum reactor? Explain your answer.
\begin{itemize}
    \item I believe this to be a thermal-spectrum reactor. There are two modes, centered at high energy ($10^5$eV), where the prompt neutrons originate, and around low energy ($10^{-1}$eV), with a dip in between where neutrons are efficiently moderated from high to low energy. This seems to correspond to the design of a thermal reactor, where neutrons are produced by fission at high energy, and moderated down to a thermal target range where fission can easily be triggered in the fissile material.
\end{itemize}
\end{enumerate}

\clearpage
\noindent 5. ({\bf flux energy spectra}) Download the file {\tt flux\_spectrum3.csv} from Canvas (in Files/Homeworks/HW2). This file contains, in CSV (comma-separated value) format, a neutron energy spectrum. On each line of this file, the data is formatted as {\tt Emin, Emax, flux}, where {\tt Emin} represents the lower bound of an energy bin, {\tt Emax} represents the upper bound of an energy bin, and {\tt flux} is the measured flux in units of 1/cm$^2$/s. 
\begin{figure}[htb!]
\centering
\includegraphics[width=0.6\linewidth]{figures/hw2_spec3_ul.png}\caption{The flux spectrum given in file \texttt{flux\_spectrum3.csv}, normalized per unit lethargy.}
\label{fig:spec3_ul}
\end{figure}
\begin{enumerate}
\item [(a)] ({\bf 5 points}) Generate a plot of flux, normalized per unit lethargy $\ln{(\text{\tt Emax/Emin})}$, as a function of energy. Include labels on your axes.
\begin{itemize}
    \item Figure~\ref{fig:spec3_ul} shows the flux spectrum given by \texttt{flux\_spectrum3.csv}, as normalized by unit lethargy.

\end{itemize}
\item [(b)] ({\bf 5 points}) What kind of nuclear system do you think this spectrum corresponds to? Explain your answer.
\begin{itemize}
    \item I believe this spectrum corresponds to a fast spectrum reactor. There is only one mode centered at extremely high energy, around 100keV, and the flux at low energies is so low that it drops off and is unrecorded.
\end{itemize}
\end{enumerate}

\clearpage
\noindent 6. ({\bf probability distribution functions and numerical integration}) $\chi_p(E)$ is a probability distribution function. The following integral represents the probability that a prompt fission neutron is born between energies $E_i$ and $E_{i+1}$.

\beq
\int_{E_i}^{E_{i+1}}\chi_p(E)dE
\eeq

An empirical formula (a curve-fit to experimental data) for $\chi_p(E)$ for \ce{^235U} thermal fission is given by the Watt spectrum\footnote{No relation to the unit of ``Watt'' -- this is named after a scientist.},

\beq
\chi_p(E)=0.4865\sinh(\sqrt{2E})e^{-E}
\eeq

where $E$ must be in units of MeV. This is not a fun integral to perform by hand. Later in the computer projects, you will use {\it numeric} integration, also called ``quadrature rules,'' to approximate integrals that are too difficult/impossible to do analytically. Numeric integration, at its most basic, seeks to approximate the area under a curve (i.e., the integral) by a sum of approximate areas. The simplest form of numeric integration uses the ``left endpoint approximation,'' illustrated below -- the integral is approximated as the sum of the eight rectangles. The accuracy of this numeric approximation improves the more intervals which are used.

\begin{figure}[htb!]
\centering
\includegraphics[width=0.3\linewidth]{figures/left_endpoint.png}
\end{figure}

\begin{itemize}
\item[] For the following items, a \texttt{python} script \texttt{integrate.py} was created and included in Appendix~\ref{sec:intscript}.
A midpoint approximation quadrature was chosen over the left endpoint.
In this quadrature, the average value over a segment was estimated using the midpoint rather than the left endpoint, but was otherwise comparable.\\
As the distribution does not have a defined upper bound, an arbitrary bound of 18MeV was chosen, by which the probability density was sufficiently low. \\
The demonstration in Figure~\ref{fig:pdf} may be found at \url{https://www.desmos.com/calculator/6mwqm4ktxj}.
\begin{figure}[htb!]
\centering
\includegraphics[width=1\linewidth]{figures/watt_desmos.png}\caption{A Desmos plot showing the integration of the given Watt probability density function, and that it drops off comfortably before the energy of 18MeV.}
\label{fig:pdf}
\end{figure}
Of note: At the given precision, the coefficient $0.4865$ may not properly normalize the distribution; an integral value of the complete function yeilds a cumulative probability of $100.5\%$, rather than the expected $100\%$.
\item[(a)] ({\bf 5 points}) Using numeric integration, find the probability that a prompt fission neutron is born between the energies of 1 MeV $\leq E\leq$ 1.1 MeV. 
\begin{verbatim}
>>> import integrate as it
>>> it.center_integral(it.watt, 1, 1.1, 1000)
0.034252746522602476
\end{verbatim}
$\rightarrow3.425\% $
\item[(b)] ({\bf 5 points}) Using numeric integration, find the probability that a prompt fission neutron is born with an energy lower than 1 keV.
\begin{verbatim}
>>> it.center_integral(it.watt, 0, 0.001, 1000)
1.4498868445903682e-05
\end{verbatim}
$\rightarrow0.01450\% $
\item[(c)] ({\bf 4 points}) Using numeric integration, find the probability that a prompt fission neutron is born with an energy higher than 3 MeV.
\\Using the aforementioned upper bound of 18MeV:
\begin{verbatim}
>>> it.center_integral(it.watt, 3, 18, 1000)
0.2134951498454279
\end{verbatim}
$\rightarrow 21.35\%$
\item[(d)] ({\bf 1 point}) What are the units of $\chi_p(E)$?
\\This function is a probability density function: the integral of $\chi_p(E)$ over energy $E$ should yield a unitless probability.
Therefore, it would have the reciprocal units of $E$, MeV$^{-1}$.
This characterizes the midpoint integration as splitting the integral domain into ``slices'' over energy, each of which is multiplied by a corresponding probability \emph{per energy} to yield a total, unitless probability. 
\end{itemize}

\clearpage
\noindent 7. ({\bf atom and weight percent review}) Consider UO$_2$ fuel with a density of 10.5 g/cm$^3$. The uranium is 4.5 wt\% (weight percent, also written as ``w/o'') \ce{^{235}U} with the remainder being \ce{^{238}U}. The oxygen is 99.8 at\% (atom percent, also written as ``a/o'') \ce{^{16} O}, with the remainder being \ce{^{18}O}. 

\begin{itemize}
\item[(a)] ({\bf 12 points}) Calculate the number density of each isotope. Be careful with the difference between atom and weight percent! Report your answer to THREE places after the decimal.\\

{\it Hint}: the number density of \ce{^235U} is between $1\times10^{21}$ atoms/cm$^3$ and $2\times10^{21}$ atoms/cm$^3$.
\begin{itemize}
    \item Allow us to first calculate the atom, or molar fraction of uranium-235 in the fuel $n_{235}$, using the relationship between mass $m$, number $n$, and molar mass $M$.
    \begin{align*}
        m &= nM \\
        n &= m/M \\
        \mathrm{at}\%_{235} &= \frac{n_{235}}{n_{235} + n_{238}} \\
                            &= \frac{m_{235}/M_{235}}{m_{235}/M_{235} + m_{238}/_{238}} \\
                            &= \frac{m_{tot}(0.045)/{235}}{m_{tot}(0.045)/M_{235} + m_{tot}(0.955)/_{238}} \\
                            &= \frac{0.045/235}{0.045/235 + 0.955/238} \\
                            &= \frac{0.0001915}{0.0001915 + 0.0042040} \\
                            &= 0.04555 \\
                            &\rightarrow\mathrm{at}\%_{235} = 0.0455 \\
    \end{align*}
    Using this, we may then find the molar mass $M_{UO_2}$ of an ``average'' molecule of UO$_2$:
    \begin{align*}
        M_{UO_2} &= M_U + 2\times M_O \\
        &= (0.04555\times235 + 0.95445\times238) + 2\times(0.998\times16 + 0.002\times18) \\
        &= 237.863 + 2\times16.004 \\
        &= 269.871\mathrm{g/mol}
    \end{align*}
    And then the number density $N$ of the overall UO$_2$ using the given mass density $\rho$:
    \begin{align*}
        N_{UO_2} &= \rho / M_{UO_2} \\
        &= \frac{10.5 \mathrm{g/cm^3} }{269.871\mathrm{g/mol}} \\
        &= 0.03891 \mathrm{mol/cm^3} \times 6.022\cdot10^23\mathrm{/mol}
        &= 2.343\cdot10^22 \mathrm{/cm^3}
    \end{align*}
    The number density of each given isotope $i$ of its element $I$ is then the atomic fraction $x_i$ of that isotope times the number of atoms per molecule $\frac{N_I}{N_{UO_2}}$ times this number density of the overall molecule:
    \begin{align*}
        N_i &= x_i * \frac{N_I}{N_{UO_2}} * N_{UO_2} \\
        N_{U235}    &= 0.04555\times 1\times2.343\cdot10^22 \mathrm{/cm^3} &= 1.067\cdot10*21/\mathrm{cm^3} \\
        N_{U238}    &= 0.95445\times 1\times2.343\cdot10^22 \mathrm{/cm^3} &= 2.236\cdot10*22/\mathrm{cm^3} \\
        N_{O16} &= 0.998\times 2\times2.343\cdot10^22 \mathrm{/cm^3} &= 4.677\cdot10*22/\mathrm{cm^3} \\
        N_{O18} &= 0.002\times 2\times2.343\cdot10^22 \mathrm{/cm^3} &= 9.372\cdot10*19/\mathrm{cm^3} \\    
    \end{align*}
\end{itemize}
\end{itemize}

\appendix
\section{Numeric Integration}
\label{sec:intscript}
The following \texttt{python} script \texttt{integrate.py} was used to perform the numeric integration in Problem~5:
\begin{verbatim}
import numpy as np

# Shortform

def sinh(x):
    return np.sinh(x)

def sqrt(x):
    return x**0.5

def exp(x):
    return np.exp(x)

# The Watt function (not the electric Watt)

def watt(E: float):
    '''
    Docstring for watt
    
    :param E: Neutron energy. Must be in units of MeV
    :type E: float
    '''
    return 0.4865 * sinh(sqrt(2*E)) * exp(-1*E)

# Integrating procedures

def center_integral(func, a:float, b:float, steps:int):
    diff = b - a
    dx = diff / steps
    sum = 0
    for i in range(steps): # i.e. excluding b itself
        left_x = a + (i+0.5)*dx
        left_val = func(left_x)
        sum += left_val * dx
    return sum
\end{verbatim}

Run from a python terminal as follows:
\begin{verbatim}
>>> import integrate as it
>>> it.center_integral(it.watt, 1, 1.1, 1000000)
0.034252746516422454
\end{verbatim}

\end{document}

