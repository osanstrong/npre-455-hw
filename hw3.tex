\documentclass[11pt]{article}
\usepackage[hidelinks]{hyperref}
\usepackage[letterpaper]{geometry}
\geometry{verbose,tmargin=1in,bmargin=1in,lmargin=1in,rmargin=1in}
\usepackage[acronym,nomain,nonumberlist,nogroupskip,nopostdot]{glossaries} % for glossary of acronyms
\usepackage{siunitx}
\usepackage{verbatim}
\usepackage{booktabs}
\usepackage{multirow}
\usepackage{threeparttable}
\usepackage{longtable}
\usepackage{rotating}
\usepackage{pdflscape}
\usepackage{cancel}
\usepackage{mathrsfs}
\usepackage{mhchem}
\setcounter{tocdepth}{4}
\setcounter{secnumdepth}{4}
\usepackage{xcolor}
\usepackage{amsmath,mathtools}
\usepackage{physics} % Vector notation
\newcommand{\hO}{\hat{\Omega}}

\setacronymstyle{long-short}

\newcommand{\mdash}{\discretionary{}{}{\kern 0.1em}---\discretionary{}{}{\kern 0.1em}}
\newcommand\Tstrut{\rule{0pt}{2.6ex}}         % = `top' strut
\newcommand\Bstrut{\rule[-0.9ex]{0pt}{0pt}}   % = `bottom' strut
\newcommand{\beq}{\begin{equation}\begin{aligned}}
\newcommand{\eeq}{\end{aligned}\end{equation}}
\newcommand{\beqs}{\begin{equation*}\begin{aligned}}
\newcommand{\eeqs}{\end{aligned}\end{equation*}}
\newcommand{\ans}[1]{{\color{violet}#1}}

\setlength{\parskip}{5pt}

\begin{document}
\noindent {\bf \Large NPRE 455: Homework 3}\\
\noindent {\bf Owen Strong \hfill Feb. 12, 2026}

\noindent 1. ({\bf factual concept recall}) Provide a {\it short} (1-2 sentences) answer to each of the following questions.

\begin{itemize}
\item[(a)] ({\bf 2 points}) If $\hat{n}$ is an outward facing normal for a control volume, describe the meaning when the neutron current vector $\vec{j}$ dotted with $\hat{n}$ is (i) positive, (ii) negative, or (iii) zero.
\item[] \ans{When the product is positive, the net current is leaving the volume at that point on the surface. When the product is negative, it is entering the volume, and when zero, it is tangential, neither entering nor exiting.}
\item[(b)] ({\bf 1 point}) If $f(x,y,z,t,E,\hat{\Omega})$ has units of $\gamma$, what are the units of the following term?

\beq
f(x,y,z)\equiv\int_{4\pi}d\hat{\Omega}\int dE\int dt\ f(x,y,z,t,E,\hat{\Omega})
\eeq
\item[] \ans{It is integrated over dimensionally significant variables $t$ and $E$, resulting in a quantity with units dimensionally equivalent to $\gamma\cdot s\cdot eV$.}
\item[(c)] ({\bf 1 point}) What is the difference between a radian and a steradian?
\item[] \ans{Where a radian describes 1D angular procession around a unit circle, a steradian describes the 2D angular coverage over a unit sphere, akin to a field of view.}
\item[(d)] ({\bf 1 point}) Explain the physical meaning of a quantity which is {\it isotropic}.
\item[] \ans{If a quantity is isotropic, then it is uniform with respect to angle. Integrating such a function over all angles is equivalent to scaling it by $4\pi$.}
\item[(e)] ({\bf 4 points}) Scalar current $\vec{J}$ and scalar flux $\phi$ have the same units. Provide a qualitative discussion of what each term represents, and describe any differences between the two quantities.
\item[] \ans{Dimensionally, both correspond to a number of crossings of unit surfaces per second, or distance traveled per unit volume per second (i.e. accounting for faster neutrons hitting things more frequently), but where scalar flux represents the ``gross'' distance traveled by neutrons regardless of direction, scalar \emph{current} is the integral of ``net'' distance traveled, where neutrons moving in opposite directions could cancel each other out. For example, in a uniform isotropic flux field, there could be gazillions of neutrons traveling at ludicrous speed (high scalar flux), but because there wouldn't be any \emph{net} procession of neutrons, the scalar current would be effectively zero.}
\item[(f)] ({\bf 1 point}) Consider a surface below, with unit normal shown. Which partial current, $J_+$ or $J_-$, is shown as the red arrow?

\begin{figure}[htb!]
\centering
\includegraphics[width=0.25\linewidth]{figures/hw3Q_pc.png}
\end{figure}
\item[] \ans{The partial current $J_-$, entering the control volume, is shown.}

\item[(g)] ({\bf 2 points}) What is the difference between $\sigma_s(E)$ and $\sigma_s(E\rightarrow E')$? Do they have the same units?
\item[]\ans{While the two are both microscopic cross-sections with the same units, $\sigma_s(E)$ represents the cross-section of a neutron with energy $E$ scattering at all, where $\sigma_s(E\rightarrow E')$ represents its cross-section to scatter to a \emph{specific energy $E'$}. $\sigma_s(E)$ would be equivalent to integrating $\sigma_s(E\rightarrow E')$ over the product energy $E'$.}
\item[(h)] ({\bf 3 points}) List at least 3 assumptions that are made when deriving the neutron transport equation. Are these very severe assumptions?
\item[] \ans{\begin{itemize}
    \item Neutrons only move in straight lines between collisions, such that only collisions change their energy and direction. This wouldn't work if, for example, they were charged like protons.
    \item We neglect free neutron decay, which has a half life of approximately 10 minutes. This works well because neutrons are really fast, so they don't take \emph{nearly} that long to hit something or leave our system.
    \item We neglect neutron-on-neutron collisions. This works well because neutrons are a lot smaller and harder to collide with than the atoms in the material.
    \item Collisions are instantaneous. This is noteworthy, but the actual times are small enough that they shouldn't change much. 
\end{itemize}
All of these assumptions are not very severe; they don't change very much from the most rigorous model.
}
\item[(i)] ({\bf 2 points}) Describe the process by which the diffusion equation is derived from the neutron transport equation. What did we have to assume about the angular flux?
\item[] \ans{The neutron diffusion equation is found by taking two moments, an integral over angle (zeroth moment) and an integral of the function times angle, over angle (first moment), but also introducing a simplifying assumption so that these moments can yield a much nicer equation. This is by far the greatest assumption which simplifies the neutron transport equation to the neutron diffusion equation: that angular flux is \emph{linearly anisotropic}, that is, while not necessarily perfectly uniform over angle, it can at least be described by a linear function of angle.}
\item[(j)] ({\bf 3 points}) List at least 3 scenarios where the diffusion equation may give inaccurate predictions.
\item[] \ans{
    The neutron diffusion equation notably fails in at least the following scenarios:
    \begin{itemize}
        \item Media without scattering, such as a vacuum.
        \item Near strong sources or sinks of neutrons: there is a strong preferential direction for neutrons which breaks our nice linear anisotropy.
        \item Strong spatial variations in general tend to introduce stronger angular dependence than our assumption permits.
        \item Near (within several mean free paths of) problematic boundaries or discontinuities, such as a vacuum, where neutrons that scatter in that direction wouldn't come back.
    \end{itemize}
}
\end{itemize}

\clearpage
\noindent 2. ({\bf angular definitions}) A mono-energetic beam of neutrons has a number density of $n$ [neutrons/cm$^{3}$/eV/sr] and velocity $v$ [cm/s] is traveling in the direction $1\hat{x}$. 

\begin{figure}[htb!]
\centering
\includegraphics[width=0.3\linewidth]{figures/hw3Q_flux.png}
\end{figure}

\begin{enumerate}
\item [(a)] ({\bf 5 points}) What is the angular current $\vec{j}$?
\item[] \ans{Since all neutrons are moving in the same direction $1\hat{x}$, the angular current is $\vec{j}=nv\hat{x}$}
\item [(b)] ({\bf 5 points}) What is the net rate at which neutrons pass through a surface of area $A$ cm$^2$ inclined at an angle $\varphi$ relative to the $x$-axis? This surface is shown as the diagonal black line above.
\item[] \ans{The net rate crossing the surface at any point with normal $\hat{n}$ (let us assume outward facing means upwards and to the left, so that current looks like it will enter the surface) as $\vec{j}\cdot\hat{n}$. If we assume the surface is flatly angled at $\varphi$, and since our current is constant, that will evaluate for all points on the surface to:
\begin{align*}
    \vec{j}\cdot\hat{n}&=nv\hat{x}\cdot\hat{n} \\
    &= nv \times \hat{n}_x \\
    &= nv\sin(\varphi)
\end{align*}
Since we assume constant density and speed in the beam over any given point on the surface, we can integrate over the surface by simply multiplying by area to obtain a rate of $Anv\sin(\varphi)$.
The area $A$, the velocity $v$, and the volumetric term of the number density $n$ cancel out to produce a rate. (Differentially to a specific energy and solid angle, though I suppose it should be clear \emph{which} angle is relevant here.)
}
\end{enumerate}

\clearpage
\noindent 3. ({\bf current and flux definitions}) A small source of \ce{^252Cf} is placed at the origin; it is so small, that we can treat it as a point. The source emits $S$ [neutrons/s] isotropically in an infinite vacuum.

\begin{itemize}
\item[(a)] ({\bf 4 points}) What is the total current, $\vec{J}(r)$, at a distance $r$ from the source?
\item[] \ans{Since the source is in an infinite vacuum, we can assume that the neutrons never change direction always facing away from the source (thus treating angular current as a delta function w.r.t. angle), 
\begin{eqnarray}
    \vec{J}(r)=\int\vec{j}(r)d\hat{\Omega}=J\hat{n}
\end{eqnarray}
We also only consider neutrons emitted by the source, such that for a control surface sphere around the point at given radius $r$:
\begin{equation}
    S=\int \vec{J}(r)\cdot\hat{n} dA
\end{equation}
Since we assume the source emits neutrons anisotropically:
\begin{eqnarray}
    S=AJ(r)\\
    J(r)=\frac{S}{A}=\frac{S}{4\pi r^2}\\
    \boxed{\vec{J}(r)=\frac{S}{4\pi r^2}\hat{n}}
\end{eqnarray}
where $\hat{n}$ is the normal vector facing away from the source.
}
\item[(b)] ({\bf 4 points}) What is the total flux, $\phi(r)$, at a distance $r$ from the source?
\item[] \ans{
    In this specific case, the \emph{only} neutrons moving around are the ones moving perfectly isotropically outwards, so the flux has the same magnitude as the total current:
    \begin{equation}
        \phi(r)=J(r)=\boxed{\frac{S}{4\pi r^2}}
    \end{equation}
}
\end{itemize}

\clearpage
\noindent 4. ({\bf current and flux definitions}) Three isotropic neutron sources, each emitting $S$ [neutrons/s], are located in an infinite vacuum at the three corners of an equilateral triangle of side length $a$. All of the neutrons have the same energy.

\begin{figure}[htb!]
\centering
\includegraphics[width=0.7\linewidth]{figures/hw3Q_star_annotated.png}
\end{figure}

\begin{itemize}
\item[(a)] ({\bf 5 points}) Find the scalar current $\vec{J}$ at the midpoint of one side (indicated with a star).
\item[] \ans{
    Following the work in Problem 3, let us consider the individual total scalar current contributions from each source:
    \begin{align*}
        \vec{J} &= \sum_{i=1}^{3}\vec{J_i} = \sum_{i=1}^{3}J_i\hat{u_i} \\
        &= J_1\hat{u_1} + J_2\hat{u_2} + J_3\hat{u_3} \\
        &\rightarrow\color{gray}J(r)=\frac{S}{4\pi r^2} \\
        &= \frac{S}{4\pi (a\cos{30})^2}\expval{\cos{30},\sin{30}} + \frac{S}{4\pi (\frac{a}{2})^2}\expval{\sin{30},-\cos{30}} + \frac{S}{4\pi (\frac{a}{2})^2}\expval{-\sin{30},\cos{30}} \\
        &= \frac{S}{4\pi a^2\frac{3}{4}}\expval{\cos{30},\sin{30}} \\
        &= \boxed{\frac{S}{3\pi a^2}\expval{\cos{30},\sin{30}} }
    \end{align*}
}
\item[(b)] ({\bf 5 points}) Find the scalar flux $\phi$ at the midpoint of one side (indicated with a star).
\item[] \ans{
    For scalar flux $\phi$, we do not consider the vector direction when adding.
    \begin{align*}
        \phi &= \sum_{i-1}^{3}\phi_i\\
        &= \frac{S}{4\pi (a\cos{30})^2} + \frac{S}{4\pi (\frac{a}{2})^2} + \frac{S}{4\pi (\frac{a}{2})^2} \\
        &= \frac{S}{3\pi a^2} +2\frac{S}{\pi a^2} \\
        &= \frac{S}{3\pi a^2} +6\frac{S}{3\pi a^2} \\
        &= \boxed{7\frac{S}{3\pi a^2}} \\
    \end{align*}
}
\end{itemize}

\clearpage
\noindent 5. ({\bf isotropy}) Consider a system with an isotropic angular flux.

\begin{enumerate}
\item[(a)] ({\bf 5 points}) Show that the scalar current is zero. 
\item[] \ans{
    \begin{align*}
        \vec{J}&=\int_{4\pi}d\hat{\Omega}\vec{j} \\
        &= \int_{4\pi}d\hat{\Omega}\psi\hat{Omega} \\
        &\color{gray}\rightarrow anisotropic \\
        &= \psi\int_{4\pi}d\hat{\Omega}\hat{\Omega}\\
        \vec{J}&= \psi\cdot\vec{0} = \vec{0}
    \end{align*}

}
\item[(b)] ({\bf 2 points}) Describe the physical meaning of a zero scalar current.
\item[] \ans{If there is zero scalar current, then there is no net direction that the neutrons are tending towards.
This makes sense for the assumption that angular flux is anisotropic and even in every direction: for any direction that neutrons would enter, an equal and opposite number would enter from the opposite direction, canceling out to zero.}
\end{enumerate}

\clearpage
\noindent 6. ({\bf diffusion approximation}) The P$_1$ approximation is equivalent to assuming that the angular flux is linearly anisotropic:

\beq
\label{eq:p1}
\psi(\vec{r},E,\hO,t)=&\ A_0(\vec{r},E,t)+B_1(\vec{r},E,t)\Omega_x+B_2(\vec{r},E,t)\Omega_y+B_3(\vec{r},E,t)\Omega_z+C_1(\vec{r},E,t)\Omega_x\Omega_x+\cdots\\
\approx&\ A_0(\vec{r},E,t)+\underbrace{B_1(\vec{r},E,t)\Omega_x+B_2(\vec{r},E,t)\Omega_y+B_3(\vec{r},E,t)\Omega_z}_{\vec{B}\cdot\hO}
\eeq

\begin{enumerate}
\item[(a)] ({\bf 5 points EXTRA CREDIT}) Derive the functions $A_0$, $B_1$, $B_2$, and $B_3$ by taking the zeroth and first moments of Eq. \eqref{eq:p1} with respect to $\hO$. (Hint: $\int_{4\pi}d\hO\Omega_i=0$ and $\int_{4\pi}d\hO\ \vec{w}\cdot\hO\hO=4\pi\vec{w}/3$.) For your final answer, write Eq. \eqref{eq:p1} with these terms inserted.
\item[] \ans{:( Aw man, I was actually kinda looking forward to this one I just ran out of time}
\item[(b)] ({\bf 10 points EXTRA CREDIT}) Derive the expression for the partial scalar current $J_+$ by inserting the P$_1$ approximation (your answer to question \#1) into the following expression for $J_+$:
\beq
\label{eq:Jp}
J_+=\int_{2\pi^+}d\hO\psi(\vec{r},E,\hO,t)\hO\cdot\hat{n}
\eeq

The $2\pi^+$ is shorthand notation to indicate we are doing the integral over the top hemisphere (for $0\leq\theta\leq\pi/2$), as opposed to the bottom hemisphere, $2\pi^-$. You do not need to perform the integrals by hand (you can use a tool such as Wolfram Alpha), but do write out the full expression and keep your work clear.
\end{enumerate}

\clearpage
\noindent 7. ({\bf deep concept understanding}) For each of the following statements, indicate whether the statement is true or false. Provide an explanation.

\begin{enumerate}
\item[(a)] ({\bf 5 points}) (True/False) The diffusion equation is the same as assuming that the angular flux is isotropic.
\item[] \ans{False. Not only is there a little more to it, but the assumption isn't that angular flux is isotropic, rather just that it is only \emph{linearly} anisotropic, and can be modeled $\psi=A+\hat{B}\hat{\Omega}$}
\item[(b)] ({\bf 5 points}) (True/False) The diffusion equation can be applied to predict neutron flux in a vacuum.
\item[] \ans{False. The diffusion equation is, well, diffusive: it relies on neutrons scattering off of nuclei. In a vacuum, that doesn't happen.}
\end{enumerate}

\clearpage
\noindent 8. ({\bf flux and current definitions, neutron balance}) Under certain circumstances, the neutron flux in a sphere of radius $R$ is given as 

\beq
\label{eq:flux}
\phi(r)=\frac{\phi_0R\sin{\left(\frac{\pi r}{R}\right)}}{r}
\eeq

where $\phi_0=13.1\times10^{13}$ [1/cm$^2$/s], $R=50$ [cm], $\Sigma_a=0.108$ [1/cm], $\Sigma_f=0.0727$ [1/cm], $D=1.1$ [cm], and $\nu=2.4$. All of the neutrons are at the same energy.

\begin{enumerate}
\item[(a)] ({\bf 5 points}) Obtain the expression for the scalar current $\vec{J}$ at a point on the surface of the sphere, assuming diffusion theory applies.
\item[(b)] ({\bf 5 points}) Plot (with a computer) $\vec{J}\cdot\hat{r}$ from $0\leq r\leq R$. Label your axes. Explain the behavior you see at $r=0$.
\item[(c)] ({\bf 5 points}) What is the rate (1/s) of neutrons which leak from the surface of the sphere?
\item[(d)] ({\bf 5 points}) What is the net number of neutrons crossing the spherical surface at $r=R/2$ per second?
\item[(e)] ({\bf 10 points}) Consider a thin spherical shell between $R/2-0.5\text{cm}<r<R/2+0.5\text{cm}$. Estimate the rate of neutrons being (i) absorbed in this shell, (ii) generated in this shell, and (iii) entering or leaving the shell through the surfaces. Carry out a neutron balance using these numbers. Be clear about your assumptions; if there is any ``discrepancy'' between sources and losses, explain why it exists and what to change about your calculation to make the discrepancy smaller.
\end{enumerate}

\clearpage
\noindent 9. ({\bf solid angle integration}) Consider a function $f(\theta,\varphi)=\theta\varphi^2$. 

\beq
\int_{4\pi}d\hat{\Omega} f(\theta,\varphi)
\eeq

\begin{itemize}
\item[(a)] ({\bf 5 points}) Evaluate the above integral.
\end{itemize}

\end{document}

